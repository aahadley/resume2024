%%%%%%%%%%%%%%%%%
% This is an sample CV template created using altacv.cls
% (v1.1.5, 1 December 2018) written by LianTze Lim (liantze@gmail.com). Now compiles with pdfLaTeX, XeLaTeX and LuaLaTeX.
%
%% It may be distributed and/or modified under the
%% conditions of the LaTeX Project Public License, either version 1.3
%% of this license or (at your option) any later version.
%% The latest version of this license is in
%%    http://www.latex-project.org/lppl.txt
%% and version 1.3 or later is part of all distributions of LaTeX
%% version 2003/12/01 or later.
%%%%%%%%%%%%%%%%

%% If you need to pass whatever options to xcolor
\PassOptionsToPackage{dvipsnames}{xcolor}

%% If you are using \orcid or academicons
%% icons, make sure you have the academicons
%% option here, and compile with XeLaTeX
%% or LuaLaTeX.
% \documentclass[10pt,a4paper,academicons]{altacv}

%% Use the "normalphoto" option if you want a normal photo instead of cropped to a circle
% \documentclass[10pt,a4paper,normalphoto]{altacv}

\documentclass[10pt,a4paper,ragged2e]{altacv}

%% AltaCV uses the fontawesome and academicon fonts
%% and packages.
%% See texdoc.net/pkg/fontawecome and http://texdoc.net/pkg/academicons for full list of symbols. You MUST compile with XeLaTeX or LuaLaTeX if you want to use academicons.

% Change the page layout if you need to
\geometry{left=1cm,right=9cm,marginparwidth=6.8cm,marginparsep=1.2cm,top=1.25cm,bottom=1.25cm}

% Change the font if you want to, depending on whether
% you're using pdflatex or xelatex/lualatex
\ifxetexorluatex
  % If using xelatex or lualatex:
  \setmainfont{Carlito}
\else
  % If using pdflatex:
  \usepackage[utf8]{inputenc}
  \usepackage[T1]{fontenc}
  \usepackage[default]{roboto}
  %\usepackage[lf, sfdefault]{gandhi}
  %\usepackage[default]{lato}
  %\usepackage[ttdefault=true]{inconsolata}
  \usepackage{inconsolata}
\fi

\input{glyphtounicode}
\pdfgentounicode=1

% Change the colours if you want to
\definecolor{Mulberry}{HTML}{72243D}
\definecolor{black}{HTML}{000000}%fe812d}
\definecolor{customgold}{HTML}{a47000}
\definecolor{EnvGreen}{HTML}{407056}
\definecolor{EnvBlue}{HTML}{1E425E}{E50914} 

\definecolor{line}{HTML}{1E425E}%fe812d}
\colorlet{heading}{EnvBlue}
\colorlet{accent}{EnvBlue}
\colorlet{emphasis}{black}
\colorlet{body}{black}

% Change the bullets for itemize and rating marker
% for \cvskill if you want to
\renewcommand{\itemmarker}{{\small\textbullet}}
\renewcommand{\ratingmarker}{\faCircle}

%% sample.bib contains your publications
%\addbibresource{sample.bib}

\usepackage{multicol}

\begin{document}
\name{Aaron Hadley}
\tagline{ Experienced engineer with experience in cryptography, systems programming, digital assets, and quantum information. \\ \textbf{Seeking to make Earth a nicer place to live.} }
\personalinfo{%
  % Not all of these are required!
  % You can add your own with \printinfo{symbol}{detail}
  \email{aahadley1@protonmail.com  }
  \location{\ Aiken, SC} \hfill\\
  \github{github.com/aahadley\ \quad \quad \quad} 
  \phone{(407) 401-3174} \hfill
  %% You MUST add the academicons option to \documentclass, then compile with LuaLaTeX or XeLaTeX, if you want to use \orcid or other academicons commands.
  % \orcid{orcid.org/0000-0000-0000-0000}
}

%% Make the header extend all the way to the right, if you want.
%\begin{fullwidth}
\makecvheader
%\end{fullwidth}

%% Depending on your tastes, you may want to make fonts of itemize environments slightly smaller
\AtBeginEnvironment{itemize}{\small}

%% Provide the file name containing the sidebar contents as an optional parameter to \cvsection.
%% You can always just use \marginpar{...} if you do
%% not need to align the top of the contents to any
%% \cvsection title in the "main" bar.

\cvsection[page1sidebar]{Work}
    \cvevent{Applied Cryptographer, Devops Developer : zEncryption}{IBM}{2022 -- }{Charlotte, NC}
\begin{itemize}
    \item Implementing and providing hardware acceleration support for new cryptographic algorithms
    \item Providing IBM Z support for fully homomorphic encryption software
    \item Applying knowledge of quantum information to aid in the development of quantum-safe cryptography
    \item Managing master key information for internal business data on \\zOS Mainframes
\end{itemize}  
    
    \divider 
    
    \cvevent{Junior Software Engineer : TransformIT}{Arxan Technologies}{2020 -- 2021}{Lafayette, IN}
\begin{itemize}
    \item Developed whitebox cryptography software for key and data protection
    \item Extended platform support and test coverage to support WatchOS and \\MacOS-arm
    \item Developed new whitebox techniques to harden against new attacks and \\penetration tests
\end{itemize}


    
    \divider 
    
    \cvevent{Member of Technical Staff : Cryptoassets, HSM}{Anchorage Digital}{2021 -- 2022}{Remote}
\begin{itemize}
    \item Developing asset support software for hardware security modules 
    \item Continually integrating new digital assets into the Anchorage banking \\platform
    \item Securing private keys for all of our customers, safeguarding billions of \\dollars worth of assets
\end{itemize}  

    
    \divider

   % \cvevent{DevOps Developer}{IBM}{2023 -- Present}{Remote}
\begin{itemize}
    \item Developing test data tools for data masking and data fabrication
    \item Working with product owners to develop effective and compliant testing strategies  
\end{itemize}
    
   % 
\cvevent{Teaching Assistant}{University of Central Florida}{January 2019 -- December 2019}{Orlando, FL}
\begin{itemize}
    \item Graded exams and programming assignments for Introduction to Programming in C and Systems Software (Introductory Compiler Design)
\end{itemize}
 


\cvsection{Research}
    
\cvevent{Quantum Information Group}{UCF Physics Department}{2018 -- 2019}{}

\begin{itemize}
    \item Participated in weekly journal clubs discussing advanced topics in Simulation of quantum systems, condensed matter physics, quantum cryptography, quantum key distribution, and linear optical quantum computing (LOQC)

\end{itemize}

    
\cvevent{Theorem Proving with Deep Reinforcement Learning}{UCF Computer Science Department}{January 2019 -- December 2019}{}
\begin{itemize}
    \item Investigated deep learning techniques for automated theorem proving software.
     \item Used Microsoft's Lean prover to interact with a deep learning interface.
    \item Extended Google's Deepmath theorem prover to be configurable via Jupyter notebook.
\end{itemize}

% TODO: Rework bullet-points to be more quantitative.


\medskip
\clearpage

%% If the NEXT page doesn't start with a \cvsection but you'd
%% still like to add a sidebar, then use this command on THIS
%% page to add it. The optional argument lets you pull up the
%% sidebar a bit so that it looks aligned with the top of the
%% main column.
%\addnextpagesidebar[-1ex]{page1sidebar}


\end{document}
